\documentclass[12pt]{article}
\usepackage{graphicx}
\usepackage{hyperref}
\usepackage[latin1]{inputenc}
\usepackage{indentfirst}
\usepackage{amsmath,amssymb}
\DeclareMathOperator*{\E}{\mathbb{E}}
\DeclareMathOperator*{\Proba}{\mathbb{P}}
\DeclareMathOperator*{\Indicator}{\mathbb{1}}

\begin{document}
\graphicspath{{/home/user/Documents/tex/assets/}}
\title{Statistically optimal v3 LP ranges for incentivized stable pools}
\author{Sety \footnote{
      \href{https://etherscan.io/address/0xFaf2A8b5fa78cA2786cEf5F7e19f6942EC7cB531}{\includegraphics[width=0.5cm]{ens.png}}
      \href{https://debank.com/profile/0xfaf2a8b5fa78ca2786cef5f7e19f6942ec7cb531}{\includegraphics[width=0.5cm]{debank.png} }
      \href{https://github.com/Sety-project}{\includegraphics[width=0.5cm]{github.png}}
      \href{https://t.me/daviidarr}{\includegraphics[width=0.5cm]{telegram.png}}
      \href{https://X.com/Sety08052022}{\includegraphics[width=0.5cm]{X.png}}
      \includegraphics[width=0.5cm]{discord.png} Sety\#2111
      }}

\date{\today}
\maketitle

\abstract We compute statstically optimal v3 LP ranges for incentivized stable pools. 

\section{Objective function: expected rewards}
We  consider a pool where a user provides liquidity $l$ in a static range $[p_l, p_u]$.

\subsection{Rewards}
\indent As per \href{https://docs.angle.money/merkl/introduction#customizable-distribution-formula}{Merkl doc} The pool is incentivized by a per block reward 
\begin{equation} 
R_t = w_f \frac{f_t}{F_t} + w_x \frac{x_t}{X_t} + w_y \frac{y_t}{Y_t}
\end{equation}
$\Rightarrow$ Let's compute the expectation of each term, and divide by the capital commited to get a ROI.
\subsection{Holdings}
User holds reserves $x_t, y_t$ out of total reserves $X_t, Y_t$. 
As per \href{https://blog.uniswap.org/uniswap-v3-math-primer-2}{uniswap blog}, when in range we have:
\[
\begin{cases}
      x_t = l \frac{\sqrt{p_u}-\sqrt{P}}{\sqrt{p_u}\sqrt{P}} \\
      y_t = l (\sqrt{P}-\sqrt{p_l})
\end{cases}
\]
$\Rightarrow$ To get the expectation of holdings, all we need to estimate is the price distribution.

\subsection{Capital}
This also implies that capital commited is:
\begin{equation}
C = y + x \cdot P = 2 l \sqrt{P} [1 - \frac{1}{2}(\frac{\sqrt{P}}{\sqrt{p_u}}+\frac{\sqrt{p_l}}{\sqrt{P}})]
\end{equation}
$\Rightarrow$ We'll use it in ROI calculation.

\subsection{Fees}
As per \href{https://whitepaper.io/document/708/uniswap-whitepaper}{uniswap whitepaper}, the fees are given by:
\begin{eqnarray*}
      feeGrowthGlobalX128 &= f^{pool} \sum{\frac{V_t}{L_t}} \\
      feeGrowthInsideX128 &= f^{pool} \sum{1_{P \in [p_l,p_u]} \frac{V_t}{L_t}} \\
      positionFees_t &= feeGrowthInsideX128 \cdot l \\
\end{eqnarray*}
$\Rightarrow$ To get the expectation of fees, all we need to estimate is the conditional expected diluted volume, and sum it over any candidate range.
\begin{equation}
\tilde{V}(i)=\E[\frac{V_t}{L_t} | P_t \in [tick(i)]]
\end{equation}


\section{Modelling}
The pool has a price $P_t$ and a diluted volume $\frac{V_t}{L_t}$ that we model as a i.i.d process of joint density $\phi(P, \tilde{V})$.
\newline \indent We further assume that the user's liquidity does not affect this dymanics.
For underlyings with more than a few months of existence, there is typically more than enough data to estimate this density over a period where stationarity is a reasonnable assumption\\
\subsection{data}
From \href{https://thegraph.com/hosted-service/subgraph/uniswap/uniswap-v3}{uniswap subgraph} we query times-series for:
\begin{itemize}
      \item $P_t$
      \item $V_t$
      \item $L_t$
\end{itemize}

\end{document}
